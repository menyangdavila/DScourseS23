\documentclass{article}


\usepackage[english]{babel}
\usepackage[letterpaper,top=2cm,bottom=2cm,left=3cm,right=3cm,marginparwidth=1.75cm]{geometry}

% Useful packages
\usepackage{placeins}
\usepackage{amsmath}
\usepackage{graphicx}
\usepackage[colorlinks=true, allcolors=blue]{hyperref}
\usepackage{indentfirst}
\usepackage{setspace}
\usepackage{booktabs}
\usepackage{siunitx}
\sisetup{text-series-to-math = true ,
propagate-math-font = true}


\title{PS 12}
\author{Mengyang Davila}
\date{May 2, 2023}

\doublespacing
\begin{document}

\maketitle
\subsection*{Question 6}
The missing rate for the variable $logwage$ is 31 percent. I believe that this variable is missing not at random, as it may be missing specifically from individuals who are still in school.
\begin{table}[htbp]
\centering
\begin{tabular}[t]{lrrrrrrr}
\toprule
  & Unique (\#) & Missing (\%) & Mean & SD & Min & Median & Max\\
\midrule
logwage & 1546 & 31 & \num{1.7} & \num{0.7} & \num{-1.0} & \num{1.7} & \num{4.2}\\
hgc & 14 & 0 & \num{12.5} & \num{2.4} & \num{5.0} & \num{12.0} & \num{18.0}\\
exper & 1932 & 0 & \num{6.4} & \num{4.9} & \num{0.0} & \num{6.0} & \num{25.0}\\
kids & 2 & 0 & \num{0.4} & \num{0.5} & \num{0.0} & \num{0.0} & \num{1.0}\\
\bottomrule
\end{tabular}
\end{table}

\subsection*{Question 7}
Using Heckman selection gives the most accurate estimate of $\hat{\beta_1} = 0.091$, compared to alternative methods such as listwise deletion or replacing missing values with the sample mean. This method allows for the inclusion of observations with missing data, thereby improving the sample size and statistical power of the analysis. 

\FloatBarrier
\begin{table}[ht]
\centering
\begin{tabular}[t]{lccc}
\toprule
  & Listwise deletion & Fill mean & Heckit\\
\midrule
(Intercept) & \num{0.834}*** & \num{1.149}*** & \num{0.446}***\\
 & (\num{0.113}) & (\num{0.078}) & \num{20.553}***\\
 &  &  & (\num{0.122})\\
 &  &  & (\num{1.111})\\
hgc & \num{0.059}*** & \num{0.036}*** & \num{-1.104}***\\
 & (\num{0.009}) & (\num{0.006}) & \num{0.091}***\\
 &  &  & (\num{0.010})\\
 &  &  & (\num{0.066})\\
union1 & \num{0.222}* & \num{0.068} & \num{-1.113}***\\
 & (\num{0.087}) & (\num{0.047}) & \num{0.186}*\\
 &  &  & (\num{0.084})\\
 &  &  & (\num{0.213})\\
college1 & \num{-0.065} & \num{-0.126}** & \num{-0.565}*\\
 & (\num{0.106}) & (\num{0.048}) & \num{0.092}\\
 &  &  & (\num{0.100})\\
 &  &  & (\num{0.227})\\
exper & \num{0.050}*** & \num{0.021}** & \num{-0.506}***\\
 & (\num{0.013}) & (\num{0.007}) & \num{0.054}***\\
 &  &  & (\num{0.012})\\
 &  &  & (\num{0.030})\\
exper\^2 & \num{-0.004}** & \num{-0.001}** & \num{-0.002}+\\
 & (\num{0.001}) & (\num{0.000}) & (\num{0.001})\\
married1 &  &  & \num{-2.275}***\\
 &  &  & (\num{0.162})\\
kids &  &  & \num{0.495}***\\
 &  &  & (\num{0.114})\\
invMillsRatio &  &  & \num{-0.695}***\\
 &  &  & (\num{0.060})\\
sigma &  &  & \num{0.696}\\
rho &  &  & \num{-0.998}\\
\midrule
Num.Obs. & \num{1545} & \num{2229} & \num{2229}\\
R2 & \num{0.038} & \num{0.020} & \num{0.092}\\
R2 Adj. & \num{0.035} & \num{0.018} & \num{0.088}\\
AIC & \num{3182.4} & \num{3808.4} & \\
BIC & \num{3219.8} & \num{3848.4} & \\
Log.Lik. & \num{-1584.189} & \num{-1897.193} & \\
F & \num{12.106} & \num{9.207} & \\
RMSE & \num{0.67} & \num{0.57} & \num{0.66}\\
\bottomrule
\multicolumn{4}{l}{\rule{0pt}{1em}+ p $<$ 0.1, * p $<$ 0.05, ** p $<$ 0.01, *** p $<$ 0.001}\\
\end{tabular}
\end{table}
\FloatBarrier

\subsection*{Question 9}
The average of each set of predicted probabilities is 0.0096, which, in my opinion, appears realistic. This could be attributed to the fact that union jobs tend to provide better pay and benefits, which can be especially appealing to women who are the primary breadwinners for their families. Additionally, union jobs may offer more predictable and stable work schedules, which can be significant for women who are juggling family responsibilities.
\end{document}