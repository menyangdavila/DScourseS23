\documentclass{article}


\usepackage[english]{babel}
\usepackage[letterpaper,top=2cm,bottom=2cm,left=3cm,right=3cm,marginparwidth=1.75cm]{geometry}

% Useful packages
\usepackage{placeins}
\usepackage{amsmath}
\usepackage{graphicx}
\usepackage[colorlinks=true, allcolors=blue]{hyperref}
\usepackage{indentfirst}
\usepackage{setspace}
\usepackage{booktabs}
\usepackage{siunitx}
\sisetup{text-series-to-math = true ,
propagate-math-font = true}


\title{PS 10}
\author{Mengyang Davila}
\date{April 18, 2023}

\doublespacing
\begin{document}

\maketitle

\subsection*{Question 9}

\begin{table}[h]
	\centering
\begin{tabular}{|l|r|l|}
	\hline
	Algorithm & Accuracy & Hyperparameters\\
	\hline
	Logit & 0.85 & penalty = 0.00\\
	\hline
	Decision trees  & 0.87 & min\_n = 10, tree\_depth = 15, cost\_complexity = 0.00 \\
	\hline
	Neural Network & 0.85 & penalty = 0.08, hidden units = 9\\
	\hline
	KNN & 0.84 & neighbors = 30\\
	\hline
	SVM & 0.86 & cost = 1, rbf\_sigma = 0.25\\
	\hline
\end{tabular}
\end{table}

Based on the accuracy results, all machine learning algorithms have shown similar accuracy levels on the testing data, with the tree model exhibiting the best out-of-sample performance. The logistic model is the most efficient in terms of time, achieving comparable accuracy, while the SVM model takes the longest.

\end{document}